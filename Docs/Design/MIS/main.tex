\documentclass[12pt, titlepage]{article}
\usepackage[utf8]{inputenc}
\usepackage{geometry}
\usepackage{amsmath}
\usepackage{tabularx}
\usepackage{xcolor}
\usepackage{booktabs}
\usepackage[normalem]{ulem}
\definecolor{darkgreen}{rgb}{0.0, 0.5, 0.0}
\geometry{margin=1.5in}

\title{SE 3XA3: Module Interface Specification\\Supreme Chess}

\author{Team \#2, The Triple Grobs
		\\ Rupinder Nagra (nagrar5)
		\\ Pesara Amarasekera (amarasep)
		\\ Jonathan Cels (celsj)
}

\date{\today}

\begin{document}

\maketitle

\newpage
\begin{table}[bp]
    \caption{\bf Revision History}
    \begin{tabularx}{\textwidth}{p{3cm}p{2cm}X}
    \toprule {\bf Date} & {\bf Version} & {\bf Notes}\\
    \midrule
    2021-03-13 & 1.0 & Created and completed document.\\
    \textcolor{red}{2021-04-11} & \textcolor{red}{2.0} & \textcolor{red}{Made changes to modules to match the system, as requirements were altered during the course of the development process.}\\
    \bottomrule
    \end{tabularx}
\end{table}

\pagenumbering{arabic}

\section*{Important Information}
    Text that has been struck through, \sout{like this}, was in the previous revision of the document and has been removed. \\
    
    Text that is colored red, \textcolor{red}{like this}, is new to revision 1, and reflects the most recent information. \\
    
    Text that is colored green, \textcolor{darkgreen}{like this}, is a comment to the TA's/professor and should not be included as a part of the content in this document. \\

\newpage

%Input Module
\newpage
\section*{Input Module}
    \subsection*{Module}
        Input
    
    \subsection*{Uses}
        Shared Data Module
    
    \subsection*{Syntax}
        \#define BOARD\_X 8\\
        \#define BOARD\_Y 8
        
        \begin{center}
            \begin{tabular}{|c|c|c|c|} 
                \hline
                Routine Names & Inputs & Outputs & Exceptions \\
                \hline
                getScreenSize & & & \\
                \hline
                setBoardSize & & & \\ 
                \hline
                onClickPress & mouseDown & seq of int & \\ 
                \hline
                onClickRelease & mouseUp & seq of int & notOnBoard \\
                \hline
                getClickPressSquare & & seq of int & noClick\\
                \hline
                getClickReleaseSquare & & seq of int & noClick\\
                \hline
            \end{tabular}
        \end{center}
    
    \subsection*{Semantics}
        \subsubsection*{State Variables}
            screenX: int \#Screen pixel width\\
            screenY: int \#Screen pixel height\\
            boardX: int \#Board pixel width\\
            boardY: int \#Board pixel height\\
            boardStart: sequence of int \#(x, y) pixel coordinates of bottom left board corner\\
            squareX: int \#Square pixel width\\
            squareY: int \#Square pixel height
            
        \subsubsection*{State Invariant}
            $|boardBounds| \leq screenX*screenY$ \\
            $|boardX| = |boardY|$ \\
            $|squareX| = |squareY|$ \\
            $|boardX| = |squareX| * BOARD\_X$ \\
            $|boardY| = |squareY| * BOARD\_Y$ \\
            
    \subsection*{Environment Variables}
        mouseDown: \#Left click press provided by the user mouse\\
        mouseUp: \#Left click release provided by the user mouse\\
        
        \subsubsection*{Assumptions and Design Decisions}
            getScreenSize is called before any other access routine, followed immediately by setBoardSize
            
    \subsection*{Access Routine Semantics}
        \subsubsection*{getScreenSize} 
            transition: screenX = screen pixel width, screenY = screen pixel height \\
            exceptions: none
        
        \subsubsection*{setBoardSize} 
            transition: boardX = board pixel width based on screen width, boardY = board pixel width
            
            based on screen height, boardStart = bottom left coordinate of the board based on 
            
            screen size. $squareX = \frac{boardX}{BOARD\_X}$. $squareY = \frac{boardY}{BOARD\_Y}$ \\
            exceptions: none
                
        \subsubsection*{onClickPress:} 
            output: tuple of (pixel value of mouse along x axis, pixel value of mouse along y axis). \\
            exceptions: none
        
        \subsubsection*{onClickRelease:} 
            output: tuple of (pixel value of mouse along x axis, pixel value of mouse along y axis). \\
            exceptions: notOnBoard, if the tuple of coordinates is not within the bounds of the board. 
            
        \subsubsection*{getClickPressSquare} 
            output: tuple of (integer from 1 to BOARD\_X, integer from 1 to BOARD\_Y) where the values represent an x and y square coordinate respectively. \\
            exceptions: noClick, if called before a click input occurs
        
        \subsubsection*{getClickReleaseSquare} 
            output: tuple of (integer from 1 to BOARD\_X, integer from 1 to BOARD\_Y) where the values represent an x and y square coordinate respectively. \\
            exceptions: noClick, if called before a click input occurs
        
%Output Module  
\newpage
\section*{Output Module}
    \subsection*{Module}
        Output
    
    \subsection*{Uses}
        Shared Data Module \\
        App Module
    
    \subsection*{Syntax}
        N/A
        
        \begin{center}
            \begin{tabular}{|c|c|c|c|} 
                \hline
                Routine Names & Inputs & Outputs & Exceptions \\
                \hline
                drawSquare & string & &  \\ 
                \hline
                drawWhiteTimer & double &  &  \\ 
                \hline
                drawBlackTimer & double &  &  \\ 
                \hline
                drawBoard & sequence of (sequence of int) &  &  \\ 
                \hline
                drawChat & sequence of (char, string) &  &  \\ 
                \hline
                drawOptions & sequence of int &  &  \\ 
                \hline
                setBackground & string  &  &  \\ 
                \hline
            \end{tabular}
        \end{center}
    
    \subsection*{Semantics}
        \subsubsection*{State Variables}
            N/A

        \subsubsection*{State Invariant}
            N/A
            
    \subsection*{Environment Variables}

        \subsubsection*{Assumptions and Design Decisions}
            N/A
            
    \subsection*{Access Routine Semantics}
    
        \subsubsection*{drawSquare}
            output: draw board square \\ 
            exceptions: none 
            
        \subsubsection*{drawWhiteTimer}
            transition: draw value of the white player's timer \\
            exceptions: none 
            
        \subsubsection*{drawBlackTimer}
            transition: draw value of the black player's timer \\
            exceptions: none 
            
        \subsubsection*{drawBoard}
            transition: use $drawSquare$ routine to build game board \\
            exceptions: none 

        \subsubsection*{drawChat}
            transition: draw strings of chat messages sent by players \\
            exceptions: none 
            
        \subsubsection*{drawOptions}
            transition: draw game options (checkmate, draw, stalemate, etc.) \\
            exceptions: none 

        \subsubsection*{setBackground}
            transition: set background colour \\
            exceptions: none 

%Board Module
\newpage
\section*{Board Module}
    \subsection*{Module}
        Board
    
    \subsection*{Uses}
        Piece Module
    
    \subsection*{Syntax}
        \#define letter ['a', 'b', 'c', 'd', 'e', 'f', 'g', 'h']
        
        \begin{center}
            \begin{tabular}{|c|c|c|c|} 
                \hline
                Routine Names & Inputs & Outputs & Exceptions \\
                \hline
                initialize &  &  & \\ 
                \hline
                getXYPosition & int & tuple of int & invalidIndex \\ 
                \hline
                isBlack & int & boolean & \\ 
                \hline
                getPosition & int & tuple of int & invalidPosition \\ 
                \hline
            \end{tabular}
        \end{center}
    
    \subsection*{Semantics}
        \subsubsection*{State Variables}
            currTurn: string \#Signifies player with the current turn
            
        \subsubsection*{State Invariant}
            N/A
            
    \subsection*{Environment Variables}
            
        \subsubsection*{Assumptions and Design Decisions}
            initialize is called before any other access routine.
            
    \subsection*{Access Routine Semantics}
    
        \subsubsection*{initialize}
            transition: currTurn is set to starting chess board state, $currTurn = `w'$ \\ 
            exceptions: none
    
        \subsubsection*{getXYPosition}
            output: tuple of ($x, y$) coordinates from board square index \\ 
            exceptions: the board index does not exist
            
        \subsubsection*{isBlack}
            output: string of piece image location \\ 
            exceptions: none
        
        \subsubsection*{getPosition}
            output: tuple of (board $letter$ index, $y$ coordinate index) \\ 
            exceptions: the board position does not exist

            
%Shared Data Module
\newpage
\section*{Shared Data Module}
    \subsection*{Module}
        Shared Data
    
    \subsection*{Uses}
        App Module \\
        Timer Module \\
        Chat Module \\
        \sout{AI Module}
    
    \subsection*{Syntax}
        \#define BOARD\_X 8\\
        \#define BOARD\_Y 8
        
        \begin{center}
            \begin{tabular}{|c|c|c|c|} 
                \hline
                Routine Names & Inputs & Outputs & Exceptions \\
                \hline
                initialize & double & & \\
                \hline
                getBoardState & & sequence of (sequence of int) & \\ 
                \hline
                getTurnState & & char & \\
                \hline
                getWhiteTimer & & double & \\ 
                \hline
                getBlackTimer & & double & \\ 
                \hline
                getChatHistory & & sequence of (char, string) & \\ 
                \hline
                setBoardState & sequence of (sequence of int) & & invalidBoardState\\ 
                \hline
                setTimerValues & double, double & & \\ 
                \hline
                addMessage & char, string & & \\ 
                \hline
                clearChat & & & \\
                \hline
            \end{tabular}
        \end{center}
    
    \subsection*{Semantics}
        \subsubsection*{State Variables}
            boardState: sequence of (sequence of int) \#Chess board state with different integers for 
            
            different pieces\\
            turnState: char \#`w' or `b', representing which player's turn it is, white or black. \\
            whiteTimer: double \#Value of the player with the white pieces' turn timer. \\
            blackTimer: double \#Value of the player with the black pieces' turn timer. \\
            chatHistory: sequence of (char, string) \#Sequence of tuples where the character is `w' or 
            
            `b' representing the player, and the string is a chat message they sent. \\
            
        \subsubsection*{State Invariant}
            $|boardState| = |boardState[i]| = BOARD\_X = BOARD\_Y$ \\
            $|chatHistory| \geq 0$
            
    \subsection*{Environment Variables}
        \subsubsection*{Assumptions and Design Decisions}
            initialize is called before any other access routine.
            
    \subsection*{Access Routine Semantics}
        \subsubsection*{initialize} 
            transition: boardState is set to starting chess board state, $turnState = `w'$, 
            
            $whiteTimer = blackTimer = \textrm{input value}$, chatHistory is set to an empty sequence. \\
            exceptions: none
            
        \subsubsection*{getBoardState} 
            output: current board state \\
            exceptions: none
        
        \subsubsection*{getTurnState} 
            output: turnState \\
            exceptions: none
        
        \subsubsection*{getWhiteTimer} 
            output: whiteTimer \\
            exceptions: none
        
        \subsubsection*{getBlackTimer} 
            output: blackTimer \\
            exceptions: none
        
        \subsubsection*{getChatHistory} 
            output: chatHistory \\
            exceptions: none
        
        \subsubsection*{setBoardState} 
            transition: updates boardState to match the given board state \\
            exceptions: invalidBoardState, if the given board state is impossible to reach
        
        \subsubsection*{setTimerValues} 
            transition: updates both whiteTimer and blackTimer to match the given timer values \\
            exceptions: none
        
        \subsubsection*{addMessage} 
            transition: adds the given message to chatHistory \\
            exceptions: none
        
        \subsubsection*{clearChat} 
            transition: sets chatHistory to an empty sequence
            exceptions: none

%Piece Module  
\newpage
\section*{Piece Module}
    \subsection*{Module}
        Piece
    
    \subsection*{Uses}
        N/A
    
    \subsection*{Syntax}
        \begin{center}
            \begin{tabular}{|c|c|c|c|} 
                \hline
                Routine Names & Inputs & Outputs & Exceptions \\
                \hline
                getPiece & array of strings & string & noPiece \\ 
                \hline
                dragPiece & object of int &  & \\ 
                \hline
            \end{tabular}
        \end{center}
    
    \subsection*{Semantics}
        \subsubsection*{State Variables}
            pieceImg: string \#Image location of piece\\
            id: string \#Unique piece ID with piece type, colour, position

        \subsubsection*{State Invariant}
            N/A
            
    \subsection*{Environment Variables}
        type: \#The piece type (PAWN, KNIGHT, BISHOP, ROOK, QUEEN, KING) \\
        colour: \#The colour of the piece

        \subsubsection*{Assumptions and Design Decisions}
            N/A
            
    \subsection*{Access Routine Semantics}
    
        \subsubsection*{getPiece}
            output: string of piece image location \\ 
            exceptions: the piece type does not exist
            
        \subsubsection*{dragPiece}
            transition: id = concatenated string of piece type, colour, and position \\
            exceptions: the piece type does not exist
            

%Game Module
\newpage
\section*{Game Module}
    \subsection*{Module}
        Game
    
    \subsection*{Uses}
        N/A
    
    \subsection*{Syntax}
        \begin{center}
            \begin{tabular}{|c|c|c|c|} 
                \hline
                Routine Names & Inputs & Outputs & Exceptions \\
                \hline
                initGame &  &  & \\ 
                \hline
                resetGame &  &  & \\ 
                \hline
                handleMove & tuple of int &  & \\ 
                \hline
                getGameResult & & string &  \\ 
                \hline
            \end{tabular}
        \end{center}
    
    \subsection*{Semantics}
        \subsubsection*{State Variables}
            currMoves: array of strings \#Shows set of moves of current game\\
            boardState: string \#Saves current board state
            
        \subsubsection*{State Invariant}
            N/A
            
    \subsection*{Environment Variables}
            
        \subsubsection*{Assumptions and Design Decisions}
            initGame is called before any other access routine.
            
    \subsection*{Access Routine Semantics}
    
        \subsubsection*{initGame}
            transition: set to default board state \\ 
            exceptions: none
    
        \subsubsection*{resetGame}
            transition: reset to default board state \\
            exceptions: none
            
        \subsubsection*{handleMove}
            transition: append move to currMoves and update board state \\ 
            exceptions: none
        
        \subsubsection*{getGameResult}
            output: string of game result (checkmate, draw, stalemate, etc.) \\
            exceptions: none
            

%App Module
\newpage
\section*{App Module}
    \subsection*{Module}
        App
    
    \subsection*{Uses}
        Game Module \\
        Board Module
    
    \subsection*{Syntax}
        \begin{center}
            \begin{tabular}{|c|c|c|c|} 
                \hline
                Routines & Inputs & Outputs & Exceptions \\
                \hline
                initialize & & &\\
                \hline
                setBoard & Board, Game & Board &  \\ 
                \hline
                %Does 
                setGame & Board, Game & Game & inValidMove \\
                \hline
            \end{tabular}
        \end{center}
    
    \subsection*{Semantics}
        \subsubsection*{State Variables}
            Board: Board Module\\
            Game: Game Module
            
        \subsubsection*{State Invariant}
            N/A
            
    \subsection*{Environment Variables}
            
        \subsubsection*{Assumptions and Design Decisions}
            \textbf{initialize} is called before any other access routine
            
    \subsection*{Access Routine Semantics}
    \subsubsection*{initialize} 
            transition: Board and Game modules are initialized to begin functions\\
            exceptions: none
            
        \subsubsection*{setBoard} 
            transition: updates the Board using the Game for representation purposes\\
            exceptions: none
        
        \subsubsection*{setGame} 
            transition: updates the Game to a new valid configuration based on
            the moves played on the Board\\
            exceptions: invalidMove, if the played move is illegal

\newpage
\section*{Chat Module}
    \subsection*{Module}
        Chat
    
    \subsection*{Uses}
        N/A
    
    \subsection*{Syntax}
        
        \begin{center}
            \begin{tabular}{|c|c|c|c|} 
                \hline
                Routines & Inputs & Outputs & Exceptions \\
                \hline
                initialize & & &\\
                \hline
                setMessage & String &  & noChatInitialized\\ 
                \hline
                %Does 
                getMessage &  & String & noChatInitialized\\
                \hline
                endChat & & & noChatInitialized\\
                \hline
            \end{tabular}
        \end{center}
    
    \subsection*{Semantics}
        \subsubsection*{State Variables}
            N/A
            
        \subsubsection*{State Invariant}
            N/A
            
    \subsection*{Environment Variables}
            
        \subsubsection*{Assumptions and Design Decisions}
            \textbf{initialize} is called before any other access routine
            
    \subsection*{Access Routine Semantics}
    \subsubsection*{initialize} 
            transition: The chat is initialized with no chat data\\
            exceptions: none
            
        \subsubsection*{setMessage} 
            transition: updates the chat with a new message\\
            exceptions: noChatInitialized, when no chat is initialized before updating the message data\\
        
        \subsubsection*{getMessage} 
            output: gets the chat data in string format\\
            exceptions: noChatInitialized, when no chat is initialized to get data from\\
        
        \subsubsection*{endChat} 
            transition: ends the chat upon request\\
            exceptions: noChatInitialized, when no chat is initialized to get end\\

%Timer Module
\newpage
\section*{Timer Module}
    \subsection*{Module}
        Timer Data
    
    \subsection*{Uses}
        N/A
        
    \subsection*{Syntax}
        \begin{center}
            \begin{tabular}{|c|c|c|c|} 
                \hline
                Routine Names & Inputs & Outputs & Exceptions \\
                \hline
                initialize & double, double & & \\
                \hline
                getTime & & double & \\
                \hline
                startTimer & & & timeUp \\
                \hline
                pauseTimer & & & \\
                \hline
            \end{tabular}
        \end{center}
    
    \subsection*{Semantics}
        \subsubsection*{State Variables}
            startValue: double \#Starting time value of the timer
            incrementValue: double \#Value added to time after every move
            currentValue: double \#Current time value of the timer
            player: char \#`w' or `b', representing which player's timer it is, white or black. \\
            
        \subsubsection*{State Invariant}
            $(incrementValue = 0) \Rightarrow |startValue| \geq |currentValue|$\\
            
    \subsection*{Environment Variables}
        \subsubsection*{Assumptions and Design Decisions}
            initialize is called before any other access routine.
            
    \subsection*{Access Routine Semantics}
        \subsubsection*{initialize} 
            transition: startValue is set to the first input representing the total time on the timer. 
            
            incrementValue is set to the second input representing the time increment added after 
            
            each move. currentValue is set equal to startValue.\\
            exception: none
            
        \subsubsection*{getTime}
            output: currentValue. \\
            exceptions: timeUp, if currentValue is 0.
        
        \subsubsection*{startTimer}
            transition: currentValue starts decreasing.
            exceptions: timeUp, if currentValue reaches 0 at any point.
        
        \subsubsection*{pauseTimer}
            transition: currentValue stops decreasing and timeIncrement is added to currentValue.
            exceptions: none            




%AI Module
\newpage
\textcolor{darkgreen}{The entire AI module has been removed. Some text may not be struck through for formatting reasons, however all content from this point on in the document has been removed.}
\section*{\sout{AI Module}}
    \subsection*{\sout{Module}}
        \sout{AI}
    
    \subsection*{\sout{Uses}}
        \sout{Game Module}
    
    \subsection*{\sout{Syntax}}
        \sout{\#define WPAWN 1\\
        \#define WBISHOP 2\\
        \#define WROOK 3\\
        \#define WKNIGHT 4\\
        \#define WQUEEN 5\\
        \#define WKING 6\\
        \#define BPAWN 11\\
        \#define BBISHOP 12\\
        \#define BROOK 13\\
        \#define BKNIGHT 14\\
        \#define BQUEEN 15\\
        \#define BKING 16\\
        \#define PawnWeight 10\\
        \#define RookWeight 50\\
        \#define BishopWeight 40\\
        \#define KnightWeight 30\\
        \#define QueenWeight 90\\
        \#define KingWeight 420\\
        \#define MobilityWeight 10\\}
        \begin{center}
            \begin{tabular}{|c|c|c|c|} 
                \hline
                \sout{Routines} & \sout{Inputs} & \sout{Outputs} & \sout{Exceptions} \\
                \hline
                \sout{initialize} & \sout{Game} & &\\
                \hline
                \sout{setBoard} & \sout{Game} & &  \\ 
                \hline
                \sout{setGame} & \sout{Game} &  & \\
                \hline
                \sout{evaluateBoard} & \sout{Game} & \sout{Integer} & \\
                \hline
                \sout{makeMove} & & \sout{(PieceVal, BoardXPos, BoardYPos)} & \\
                \hline
            \end{tabular}
        \end{center}
    
    \subsection*{\sout{Semantics}}
        \subsubsection*{\sout{State Variables}}
            \sout{internalBoard: sequence of sequence of integers\\
            GameTree: Tree of Boards\\}
            % PieceVal: {WPAWN, WBISHOP, WROOK, WKNIGHT, WKING, WQUEEN,
            %           BPAWN, BBISHOP, BROOK, BKNIGHT, BKING, BQUEEN}\\
            % PieceWeight: {PawnWeight, RookWeight, BishopWeight, KnightWeight, KingWeight, QueenWeight}\\
            
        \subsubsection*{\sout{State Invariant}}
            \sout{N/A}
            
    \subsection*{\sout{Environment Variables}}
            
        \subsubsection*{\sout{Assumptions and Design Decisions}}
            \sout{\textbf{initialize} is called before any other access routine}
            
    \subsection*{\sout{Access Routine Semantics}}
    \subsubsection*{\sout{initialize} }
            \sout{transition: internal Board representation (internalBoard) initialized to begin functions}\\
            
        \subsubsection*{\sout{setBoard} }
            \sout{transition: updates internal Board using the external Game state for representation purposes\\
            exceptions: none}
        
        \subsubsection*{\sout{setGame} }
            \sout{transition: updates the internalBoard to a new valid configuration based on the moves played (the new Game state)\\
            exceptions: none}
        
        \subsubsection*{\sout{evaluateBoard}}
            \sout{output: evaluates the board according to the defined weights
            and cost function and returns $eval$ (defined below).}
            \begin{center}
            \begin{multline*}
                materialScore = KingWeight (wK-bK) + QueenWeight  (wQ-bQ)\\
                           + RookWeight (wR-bR)+ KnightWeight(wN-bN)\\
                           + BishopWeight (wB-bB) + PawnWeight (wP-bP)\\
            \end{multline*}
            \begin{gather*}
              mobilityScore = mobilityWeight (wMobility-bMobility)\\
              eval = (mobilityScore+materialScore)whoMoved
            \end{gather*}
            \end{center}
            where $wK = \# of White Kings$, $wQ = \# of White Queens$,
            $wP = \# of White Pawns$,\\ $wB = \# of White Bishops$, $wR = \# of White Rooks$, $wN = \# of White Knights$\\
            $bK = \# of Black Kings$, $bQ = \# of Black Queens$,
            $bP = \# of Black Pawns$,\\ $bB = \# of Black Bishops$, $bR = \# of Black Rooks$, and $bN = \# of Black Knights$.\\
            
           The mobility of a side is defined by the number of squares reachable by the pieces of a side. $wMobility = \# squares reachable by white$ and $bMobility = \# squares reachable by black$.\\
            
            The $whoMoved$ value is 1 if the current player is white and -1 if the current player is black.
            exceptions:none
        
        \subsubsection*{\sout{makeMove}}
        \sout{output: Evaluate the current game state and makes a move, returns a tuple that represents the piece being moved, and the position on the Board where it should be moved to.
        exceptions: none}
            
\end{document}
