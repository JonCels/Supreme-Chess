\documentclass{article}
\usepackage[utf8]{inputenc}
\usepackage{enumitem}
\usepackage[a4paper, total={6in, 8in}]{geometry}
\usepackage[normalem]{ulem}
\usepackage{booktabs}
\usepackage{tabularx}
\usepackage{hyperref}
\hypersetup{
    colorlinks,
    citecolor=black,
    filecolor=black,
    linkcolor=red,
    urlcolor=blue
}

\title{3XA3 Problem Statement, Group 2, L03}
\author{Group Name: The Triple Grobs \\ Jonathan Cels (celsj), \\Rupinder Nagra (nagrar5), \\Pesara Amarasekera (amarasep)}
\date{January 2021}


\begin{document}
\maketitle
\section{Introduction}
\subsection{What problem are you trying to solve?}
\begin{enumerate}[label={}]
    \item Now more than ever, video games are a huge source of entertainment for many people around the world. Our goal is to develop a fun and quick way to play the game of chess with friends or other users, without requiring the users to create accounts and add friends, which is required on most other platforms. Additionally, due to the recent influx of new chess players, there is high demand for learning how to play chess. This project will be implemented as a web application, offering users an alternative place to play chess.
\end{enumerate}

\subsection{Why is this problem important?}
\begin{enumerate}[label={}]
    \item Chess has suddenly risen in popularity over the last year. This can be attributed to people spending more time at home due to the pandemic, and the popularity of "The Queen's Gambit", a Netflix show that follows fictional chess prodigy Beth Harmon. People require entertainment, especially in these times, and our application aims to allow people to learn and play chess without the need for account creation. \sout{Playing with a friend requires nothing more than a link to the web server, making game creation as easy as possible.} \textcolor{red}{Playing with a friend is as easy as connecting to an online chat room and making moves, making game creation as easy as possible.}
\end{enumerate}

\subsection{What is the context of the problem?}
\begin{enumerate}[label={}]
    \item Chess is a game that is hundreds of years old, and has been extremely popular throughout its lifetime. It is able to be played by people of all ages, and we aim to continue spreading chess through the world. This is why our implementation of chess will be accessible to anyone, as nearly every device has access to a web browser and is able to run a JavaScript web application. Users will be able to play for free against friends without having to go through any hassle such as account creation or downloads, \textcolor{red}{and will be able to chat online with any other users.}
    
    \item This implementation affects multiple stakeholders, namely the developers, client, and users. The developers of this project will be us, the students implementing and maintaining it throughout its lifespan. The clients of this project are the professor and TA's of the course. The users are the players who will be able to play the game after it is deployed.
\end{enumerate}
\end{document}
